% Some commands used in this file


\chapter{\"Ozet}
Robot İşletim Sistemi (Robot Operating System, ROS), on yıldan daha kısa bir süre içerisinde robotik programlamada göz ardı edilemeyecek bir gelişim göstermiştir. ROS’un bu başarısının altında sunduğu çeşitli özellikler ve işlevselliği bulunmaktadır. İTÜ Robotik Laboratuvarı’nda da ROS’un bu avantajları fark edilerek, birçok sistemde ROS’a geçiş yapılmıştır. Laboratuvarda İTÜ-AGV ismi verilen iki adet mobil robot önceki yıllarda yapılmıştır. Bu robotlara ileride karmaşık projelere temel oluşturacak önemli uygulamaların ROS ortamında yazılması ihtiyacı bulunmaktadır. Bu proje kapsamında, ROS ile haberleşmeye uygun gömülü yazılımın ve ROS üzerinde çeşitli uygulamaların geliştirilmesi yer almaktadır.